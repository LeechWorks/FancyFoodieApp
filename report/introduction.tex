\chapter{Introduction}


\section{Scope of Project} % (fold)
\label{sec:scope_of_project}
Mobile applications(apps) are making our life more and more interesting than ever before. A lot of young people love to take photos with their smart phone, make comments to it and share with their friends about what they ate. I personally find it would be useful if there is a tool for anyone to create food event and share events with friends.\\

The aim of the project is to be able to review food. Before you start eating, a user would take a picture of the food they are going to eat and give some basic information like location, tags, date etc. Tags are used to describe the type of the food such as ``Chinese, Bun'' or ``Tea, Classic, Jasmine Green''. After finish eating, the user would make comments to the cuisine. This app also lets the user search nearby location or location defined by the user, which will show a list of pins on the map indicating the user had been there before, to decide what the user want to eat. When the user saves a food event, he or she also have the ability to share the review and photo of food to Facebook, Twitter or to an email address. For the user to see the history records, the app provides a way to view statistics data which answers to questions like how many places the user has been, how the rate is, how many tags the user used etc. ?
 
% section scope_of_project (end)

\section{Motivation} % (fold)
\label{sec:motivation}
Seeking for places to eat is a thing in my genuine. I always want to keep track of my food adventures so that next time I could have better sense of what kind of cuisine I should order. I could also give recommendations to my friends about the adventures. Luckily, it turns out that I am not the only person who want to do such kind of thing. A few of my friends tell me that people nowadays love to take photos of the food before they eat, and post their photos to all kinds of social media such as Facebook, Twitter, Weibo (Chinese Twitter) etc.. Especially for Asian students, they have really  a strong motivation to take pictures of food before they eat. \\

After searching the apps available in apple app store, I couldn't find any suitable choice for my need. So I come up with an idea that I need to make best use of skills and build a good App for people who like to explore food world with friends. \\

Currently, various of apps related to food are available in Apple's app store. But most of them focused on the whole store/restaurant review like \emph{Yelp}. This could be inaccurate sometimes because you might just hate one cuisine so you hate the restaurant. Things can be totally opposite if you love one cuisine but still hate the restaurant. So I'm hoping we could have an app only evaluate the cuisine itself. And there are also a few recipe-related apps like \emph{Evernote Food}. Food app from Evernote did a good job showing the meals. The user interface is really friendly. Adding tags to the meal is easy and choosing place information is convenient in Food. The app also provided functionality to add a cuisine recipe as well. But people who like to take pictures of their food are not always into making food. So in our project, we won't do any recipe recording. We want to have a better tool to publish what you eat instead of how that cuisine is made.
% section motivation (end)

\section{Deliverable} % (fold)
\label{sec:deliverable}
An iOS app called \emph{Fancy Foodie} is built and submitted to Apple's app store via iTunes connect. The version 1.0 is released. The category of the app are ``Food & Drinks'' and ``Lifestyle''. The default language for \emph{Fancy Foodie} is English. Another web site( http://soleo.github.io/Fancy/ ) is built to support the app.
% section deliverable (end)

\section{Intended Audience} % (fold)
\label{sec:intended_audience}
The target users for \emph{Fancy Foodie} is people who love to try different kind of food and enjoy using social media to publish their experience of food adventures.
% section intended_audience (end)

\section{Platform for Deployment} % (fold)
\label{sec:platform_for_deployment}
\begin{itemize}
\item Operating System: iOS 6.0 or later
\item Hardware: iPhone 4, 4S , 5 or iPod Touch 4th Generation
\end{itemize}
% section platform_for_deployment (end)

\section{Development Platform} % (fold)
\label{sec:development_platform}
\begin{itemize}
\item Operating System: Mac OS X 10.8.3 
\item Processor: 2.9 GHz Intel Core i7
\item Memory:  8 GB 1600 MHz DDR3
\item Integrated Development Environment(IDE): Xcode 4.6 with Automatic Reference Counting(ARC)
\item Version Control: git
\end{itemize}
% section development_platform (end)

\section{Road map} % (fold)
\label{sec:roadmap}
Fancy Foodie 1.0
\begin{description}
	\item[Core Features] \hfill 
		\begin{itemize}
		    \item Taking picture of food, storing all meta data in local database 
		    \item Listing all the food events in a table view and detailed view
			\item Showing basic statistics information of the events
			\item Searching by using current user location or user input address
			\item Configuration of fancy foodie
		 \end{itemize}
	\item[User Interface Design] \hfill 
		\begin{itemize}
		    \item Designing app logo, navigation bar, icons and background.  
		    \item Using Font-Awesome and Droid Sans font as default in app
		 \end{itemize}
	 \item[Testing] \hfill 
		\begin{itemize}
		    \item Using TestFlight do beta testing on the fly 
		    \item Testing in iPhone 5, iPod Touch 4th generation
		 \end{itemize}
\end{description}
% section road map (end)



