\chapter{Introduction}


\section{Scope of Project} % (fold)
\label{sec:scope_of_project}
Mobile Apps are making our life more and more interesting than ever before. A lot of young people love to take photo with their smart phone, making comments to it and sharing with their friend about what they had done. I personally find it would be useful if there is a tool for anyone to create food event and share with friends.\\

The particular aim of the project is to be able to review food and sharing food event with friends. Before you start eating, the user would take a picture of the food they are going to eat and give some basic information about location, tags, date etc. Tags are used to describe the type of the food such as ``Chinese, Bun'' or ``Tea, Classic, Jasmine Green''. After finish eating, the user would make comments of the food. This application also lets user search nearby location or location defined by user, which will show a list of pins on the map where the user had been there before, to decide what the user want to eat. When the user saves a food event, he or she may also share the review and foodie's photo to Facebook or Twitter or to an email address. For user to see the history records, the application also provide a way to view statistics data such as how many places the user has been, how the rates are, how many tags the user used etc. ?
 
% section scope_of_project (end)

\section{Motivation} % (fold)
\label{sec:motivation}
Seeking for places to eat is a thing in my genuine. I always want to keep track of my food adventures so that next time I could have better sense of what kind of cuisines I should order. I could also give recommendations to my friends about the adventures. Luckily, it turns out that I am not the only one who want to do such kind of thing. A few of my friends tell me that people nowadays love to take photos of the food before they eat, and post their photos to all kinds of social media such as Facebook, Twitter, Weibo (Chinese Twitter) etc.. Especially asian students have really strong motivation to take pictures of food when they eat. \\

After searching the apps available in apple app store, I didn't find any suitable choice for my need. So I came up with an idea that I need to make best use of skills and build a good app for people who like to explore food world with friends. \\

Currently, various of apps related to foodie are available in apple's app store. But most of them focused on the whole store/restaurant review. This could be inaccurate sometimes because you might just hate one dish. And there are also a few recipe-related apps. But the project want to have a better tool to publish what you eat instead of how that dish is made.
% section motivation (end)

\section{Deliverable} % (fold)
\label{sec:deliverable}
An iOS app called \emph{Fancy Foodie} is built and submitted to apple's app store via iTunes connect. The version 1.0 is released. The category of the app are ``Food & Drinks'' and ``Lifestyle''. The default language for \emph{Fancy Foodie} is English. Another website( http://soleo.github.io/Fancy/ ) is built to support the app.
% section deliverable (end)

\section{Intended Audience} % (fold)
\label{sec:intended_audience}
The target users for \emph{Fancy Foodie} is people who love to try different kind of food and enjoy using social media to publish their experience of food adventures.
% section intended_audience (end)

\section{Platform for Deployment} % (fold)
\label{sec:platform_for_deployment}
\begin{itemize}
\item Operating System: iOS 6.0 or later
\item Hardware: iPhone 4, 4S , 5 or iPod Touch 4th Generation
\end{itemize}
% section platform_for_deployment (end)

\section{Development Platform} % (fold)
\label{sec:development_platform}
\begin{itemize}
\item Operating System: Mac OS X 10.8.3 
\item Processor: 2.9 GHz Intel Core i7
\item Memory:  8 GB 1600 MHz DDR3
\item IDE: Xcode 4.6 with ARC
\item Version Control: git
\end{itemize}
% section development_platform (end)

\section{Roadmap} % (fold)
\label{sec:roadmap}
Fancy Foodie 1.0
\begin{description}
	\item[Core Features] \hfill 
		\begin{itemize}
		    \item Taking picture of food, stored all meta data in local database 
		    \item Listing all the food events in a table view and detailed view
			\item Showing basic statistics information of the events
			\item Searching by using current user location or user input address
			\item Configuration of fancy foodie
		 \end{itemize}
	\item[User Interface Elements] \hfill 
		\begin{itemize}
		    \item Created App Logo, Navigation Bar, Icons and Backgrounds.  
		    \item Used FontAwesome and Droid Font as default in App
		 \end{itemize}
	 \item[Testing] \hfill 
		\begin{itemize}
		    \item Using TestFlight do beta Testing 
		    \item Tested in iPhone 5, iPod Touch 4th Generation
		 \end{itemize}
\end{description}
% section roadmap (end)



